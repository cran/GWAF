\HeaderA{lme.batch.imputed}{function to test associations between a continuous trait and a batch of imputed SNPs in families using 
Linear Mixed Effects model}{lme.batch.imputed}
\begin{Description}\relax
Fit linear mixed effect model to test association between a continuous phenotype 
and all imputed SNPs in a genotype file. The SNP genotype is treated
as fixed effects, and a random effect correlated according to degree of relatedness within a 
family is also fitted. In each trait-SNP assocaition test, the \code{lmekin()} function which is modified from
the same named function in package \code{kinship} is used.
\end{Description}
\begin{Usage}
\begin{verbatim}
lme.batch.imputed(phenfile, genfile, pedfile, phen, kinmat, covars = NULL, outfile)
\end{verbatim}
\end{Usage}
\begin{Arguments}
\begin{ldescription}
\item[\code{phenfile}] a character string naming the phenotype file for reading (see format requirement in details) 
\item[\code{genfile}] a character string naming the genotype file for reading (see format requirement in details) 
\item[\code{pedfile}] a character string naming the pedigree file for reading (see format requirement in details)
\item[\code{phen}] a character string for a phenotype name in phenfile  
\item[\code{kinmat}] a character string naming the file where kinship coefficient matrix is kept 
\item[\code{covars}] a character vector for covariates in phenfile 
\item[\code{outfile}] a character string naming the result file for writing 
\end{ldescription}
\end{Arguments}
\begin{Details}\relax
Similar to the details for 'lme.batch' function but here the SNP data contains imputed genotypes (allele dosages) 
that are continuous and range from 0 to 2. In addition, the user 
specified genetic model argument is not available.
\end{Details}
\begin{Value}
No value is returned. Instead, results are written to \code{outfile}.

\begin{ldescription}
\item[\code{phen }] phenotype name
\item[\code{snp }] SNP name
\item[\code{N }] the number of individuals in analysis
\item[\code{AF }] imputed allele frequency of coded allele
\item[\code{h2q }] the portion of phenotypic variation explained by the SNP
\item[\code{beta }] regression coefficient of SNP covariate
\item[\code{se }] standard error of \code{beta}
\item[\code{pval }] p-value of the chi-square statistic
\end{ldescription}
\end{Value}
\begin{Author}\relax
Qiong Yang <qyang@bu.edu> and Ming-Huei Chen <mhchen@bu.edu>
\end{Author}
\begin{References}\relax
kinship package: mixed-effects Cox models, sparse matrices, and modeling data from large pedigrees.
Beth Atkinson (atkinson@mayo.edu) for pedigree functions.Terry Therneau (therneau@mayo.edu) for all other functions.
2007. Ref Type: Computer Program http://cran.r-project.org/. 

Abecasis, G. R., Cardon, L. R., Cookson, W. O., Sham, P. C., \& Cherny, S. S. Association analysis in 
a variance components framework. \emph{Genet Epidemiol}, \bold{21} Suppl 1, S341-S346 (2001).
\end{References}

