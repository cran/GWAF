\HeaderA{gee.lgst.imputed}{function for testing association between a dichotomous trait and an imputed SNP in family data using GEE}{gee.lgst.imputed}
\begin{Description}\relax
Fit logistic regression via GEE to test association between a dichotomous phenotype 
and one imputed SNP in a genotype file. Each family is treated as 
a cluster, with independence working correlation matrix used in the robust variance estimator.
This function is called in \code{gee.lgst.batch.imputed} function to apply association test to all imputed SNPs in the 
genotype data.
\end{Description}
\begin{Usage}
\begin{verbatim}
gee.lgst.imputed(snp, phen, test.dat, covar = NULL)
\end{verbatim}
\end{Usage}
\begin{Arguments}
\begin{ldescription}
\item[\code{snp}] imputed genotype data of a SNP
\item[\code{phen}] a character string for a phenotype name in phenfile 
\item[\code{test.dat}] the product of merging phenotype, genotype and pedigree data, should be ordered by "famid" 
\item[\code{covar}] a character vector for covariates in phenfile 
\end{ldescription}
\end{Arguments}
\begin{Details}\relax
Similar to the details for \code{gee.lgst} function but here the SNP data contains imputed genotypes (allele dosages) 
that are continuous and range from 0 to 2. In addition, the user specified genetic model argument is not available.
\end{Details}
\begin{Value}
Please see output in \code{gee.lgst.batch.imputed}.
\end{Value}
\begin{Author}\relax
Qiong Yang <qyang@bu.edu> and Ming-Huei Chen <mhchen@bu.edu>
\end{Author}
\begin{References}\relax
Liang, K.Y. and Zeger, S.L. (1986)
Longitudinal data analysis using generalized linear models.
\emph{Biometrika}, \bold{73} 13--22. 

Zeger, S.L. and Liang, K.Y. (1986)
Longitudinal data analysis for discrete and continuous outcomes.
\emph{Biometrics}, \bold{42} 121--130.

Vincent J Carey.Ported to R by Thomas Lumley (versions 3.13 and 4.4) and Brian Ripley (version 4.13). gee: Generalized Estimation Equation solver. 
[4.13]. 2007. Ref Type: Computer Program, http://cran.r-project.org/
\end{References}
\begin{SeeAlso}\relax
\code{gee()} function from package \code{gee}
\end{SeeAlso}

