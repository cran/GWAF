\HeaderA{gee.lgst.batch}{function to test genetic association between a dichotomous trait and a batch of genotyped SNPs in families using 
GEE}{gee.lgst.batch}
\begin{Description}\relax
Fit logistic regression via GEE to test association between a dichotomous phenotype 
and all genotyped SNPs in a genotype file with user specified genetic model. Each pedigree is treated as 
a cluster, with independence working correlation matrix used in the robust variance estimator.
This function applies the same trait-SNP association test to all SNPs in the genotype data. 
The trait-SNP association test is carried out by \code{gee.lgst} function where the 
the \code{gee()} function from package \code{gee} is used.
\end{Description}
\begin{Usage}
\begin{verbatim}
gee.lgst.batch(genfile, phenfile, pedfile, outfile, phen, covars=NULL, 
model="a", col.names=T, sep.ped=",", sep.phe=",", sep.gen=",")
\end{verbatim}
\end{Usage}
\begin{Arguments}
\begin{ldescription}
\item[\code{genfile}] a character string naming the genotype file for reading (see format requirement in details) 
\item[\code{phenfile}] a character string naming the phenotype file for reading (see format requirement in details) 
\item[\code{pedfile}] a character string naming the pedigree file for reading (see format requirement in details) 
\item[\code{outfile}] a character string naming the result file for writing 
\item[\code{phen}] a character string for a phenotype name in phenfile 
\item[\code{covars}] a character vector for covariates in phenfile 
\item[\code{model}] a single character of 'a','d','g', or 'r', with 'a'=additive, 'd'=dominant, 'g'=general and 'r'=recessive models 
\item[\code{col.names}] a logical value indicating whether the output file should contain column names 
\item[\code{sep.ped}] the field separator character for pedigree file 
\item[\code{sep.phe}] the field separator character for phenotype file 
\item[\code{sep.gen}] the field separator character for genotype file 
\end{ldescription}
\end{Arguments}
\begin{Details}\relax
The \code{gee.lgst.batch} function first reads in and merges phenotype-covariates, genotype 
and pedigree files, then tests the association of \code{phen} against all SNPs in \code{genfile}.
\code{genfile} contains unique individual id and genotype data, with the column names being "id" and SNP names.
For each genotyped SNP, the genotype data should be coded as 0, 1, 2 indicating the numbers of coded 
alleles. The SNP names in genotype file should not have any 
dash, '-' and other special characters (dots and underscores are OK). \code{phenfile} contains unique individual id, 
phenotype and covariates data, with the column names being "id" and phenotype and 
covaraite names. \code{pedfile} contains pedigree informaion, with the column names being 
"famid","id","fa","mo","sex". In all files, missing value should be an empty space, except missing parental id in \code{pedfile}.
Only phenotypes with two categories are analyzed. A phenotype should be coded as 
0 and 1, with 1 denoting affected and 0 unaffected. SNPs with low genotype counts 
(especially minor allele homozygote) may be omitted or analyzed with dominant model or
analyzed with logistic regression. 
The \code{gee.lgst.batch} function fits Generalized Estimation Equation (GEE) model using each pedigree as a cluster 
with \code{gee.lgst} function from \code{GWAF} package and 'gee' function from \code{gee} package.
\end{Details}
\begin{Value}
No value is returned. Instead, results are written to \code{outfile}.
When the genetic model is 'a', 'd' or 'r', the result includes the following columns.
When the genetic model is 'g', \code{beta} and \code{se} are replaced with \code{beta10},
\code{beta20},\code{beta21},\code{se10},\code{se20},\code{se21} .

\begin{ldescription}
\item[\code{phen }] phenotype name
\item[\code{snp }] SNP name
\item[\code{n0 }] the number of individuals with 0 copy of minor alleles
\item[\code{n1 }] the number of individuals with 1 copy of minor alleles
\item[\code{n2 }] the number of individuals with 2 copies of minor alleles
\item[\code{nd0 }] the number of individuals with 0 copy of minor alleles in affected sample
\item[\code{nd1 }] the number of individuals with 1 copy of minor alleles in affected sample
\item[\code{nd2 }] the number of individuals with 2 copies of minor alleles in affected sample
\item[\code{miss.0 }] Genotype missing rate in unaffected sample 
\item[\code{miss.1 }] Genotype missing rate in affected sample 
\item[\code{miss.diff.p }] P-value of differential missingness test between unaffected and affected samples 
\item[\code{beta }] regression coefficient of SNP covariate
\item[\code{se }] standard error of \code{beta}
\item[\code{chisq }] Chi-square statistic for testing \code{beta} not equal to zero
\item[\code{df }] degree of freedom of the Chi-square statistic
\item[\code{model }] model actually used in the analysis
\item[\code{remark }] warning or additional information for the analysis, 'not converged' indicates the 
GEE analysis did not converge; 'logistic reg' indicates GEE model is replaced by logistic regression;
'exp count<5' indicates any expected count is less than 5 in phenotype-genotype table; 'not converged 
and exp count<5', 'logistic reg \& exp count<5' are noted similarly; 'collinearity' indicates collinearity
exists between SNP and some covariates
\item[\code{pval }] p-value of the chi-square statistic
\item[\code{  }] 
\item[\code{beta10 }] regression coefficient of genotype with 1 copy of minor allele vs. that with 0 copy
\item[\code{beta20 }] regression coefficient of genotype with 2 copy of minor allele vs. that with 0 copy
\item[\code{beta21 }] regression coefficient of genotype with 2 copy of minor allele vs. that with 1 copy
\item[\code{se10 }] standard error of \code{beta10}
\item[\code{se20 }] standard error of \code{beta20}
\item[\code{se21 }] standard error of \code{beta21}
\end{ldescription}
\end{Value}
\begin{Author}\relax
Qiong Yang <qyang@bu.edu> and Ming-Huei Chen <mhchen@bu.edu>
\end{Author}

